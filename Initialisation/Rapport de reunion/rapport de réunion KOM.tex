\documentclass[]{article}

\usepackage[utf8]{inputenc}
\usepackage[french]{babel}

\title{Rapport de réunion de lancement}
\author{
    Théo Delmas\\
    Lauric Teysseyre\\
    Pierre-Louis Renon\\
    Julien Wattier\\
    \\
    Université Paul Sabatier\\
    Master Informatique 1\\
   } 
\date{21/12/2022}


\begin{document}
    \maketitle
    \newpage
    \tableofcontents
    \newpage

    \begin{section}{Participants}
        Les participants de cette réunion étaient les suivants : 
        \begin{itemize}
            \item Pablo \textsc{Seban}, maître d'ouvrage.
            \item Léo \textsc{Cusseau}, assistant à la maîtrise d'ouvrage.
            \item Florian \textsc{Azizen}, assistant à la maîtrise d'ouvrage.
            \item Théo \textsc{Delmas}, fournisseur.
            \item Lauric \textsc{Teysseyre}, fournisseur.
            \item Pierre-Louis \textsc{Renon}, fournisseur.
            \item Julien \textsc{Wattier}, fournisseur.
        \end{itemize}
    \end{section}

    \begin{section}{Objectif de la réunion}
        L'objectif de cette réunion était de confirmer la note de cadrage du projet produite en amont par l'équipe des 
        fournisseurs.\\
        Pour rappel la note de cadrage clarifie le projet, son contexte et son besoin. Elle explicite également le cadre du 
        projet, précise la gestion du projet et propose une analyse préliminaire de risques.
    \end{section}

    \begin{section}{Observations du client et ses assistants}
        \begin{subsection}{Connaissance de l'agilité}
            Florian Azizen nous a demandé si nous n'étions pas formé aux méthodes agiles avec le master, notamment parceque 
            durant la présentation nous disons qu'il y a de bonne chance d'avoir des retard sur jalon faute d'expérience et 
            qu'en cas d'absence temporaire du client nous reportons les validations.

            Nous lui avons répondu que la formation se fait en même temps que la tenu du projet. D'autre part il nous a précisé 
            qu'en méthode agile, certains rôle permettent de simuler le client lors des backlogs produit si ce dernier n'est pas 
            présent. Finalement il nous a proposé de nous envoyer un document clarifiant tous ces rôles.
        \end{subsection}
        
        \begin{subsection}{Expérience avec VueJS}
            Pablon Seban était étonné que nous évaluions peu problable le risque de manque de maîtrise sur VueJs.

            Nous avons clarifié le fait que Pierre-Louis Renon et Julien Wattier dispose d'une expérience avec Vue, ce qui 
            réduit ce risque d'occurence.
        \end{subsection}

        \begin{subsection}{Clarification de besoin}
            Pablo Seban a clarifié le besoin de documenter les contraintes en précisant qu'en fait, il vallait mieux que nous 
            nous concentrions sur les autres tâches et qu'il produirait quelques contraintes documentées selon le template afin 
            que nous puissions confortablement tester les interfaces.

            Ce problème venait d'un quiproquo de notre part car nous avions bien compris que nous ne documenterions pas 
            totalement les contraintes, mais qu'il faudrait au moins toutes les conformer au template qui serait produit.\\
            De fait ce besoin est supprimer et sera éventuellement rajouter vers la fin du projet selon l'avancement.
        \end{subsection}

        \begin{subsection}{Abus de langage sur la gestion de version}
            Lors de la présentation, pour parler de l'outil de gestion de version nous parlions de github. Or le projet est 
            gérer via framagit, ce qui a interpeler Pablo Seban qui ne comprenanait pas bien.

            Nous avons clarifier nos propos en précisant que c'était un abus de langage et qu'en fait nous parlions de git, et 
            que l'hébergeur de version serait bel et bien framagit.
        \end{subsection}

        \begin{subsection}{Gestion de version}
            Pablo Seban nous a demandé si nous allions travailler depuis des branches du dépot du projet, ou si nous allions
            forker le projet, et de manière générale de préciser le workflow.

            Après discussion avec tous les participants, il a été décidé d'appliquer le workflow github comme c'était prévu, 
            mais en considérant la branche \emph{Catalogue} comme master, permettant à tous les contributeurs du projet de 
            pouvoir suivre simplement le projet. Nous avons également présenter le workflow github.
        \end{subsection}

        \begin{subsection}{Durée des sprints absente}
            Léo Cusseau nous a demande pourquoi la durée des sprint n'était évoqué nul part.

            Nous lui avons dit que dans la phase d'initialisation du projet, il était compliqué de se prononcer sur cette durée 
            puisque à ce stade, nous avons une vision trop macro du projet. Dans la phase suivante qu'est la plannification, 
            l'étude détaillé du besoin nous permettra d'affiner suffisament les tâches à réaliser et donc le planning pour 
            décider d'une durée de sprint pertinente.
        \end{subsection}

        \begin{subsection}{Utilisation d'un kanban}
            Pablo Seban nous a demandé si c'était la peine d'utiliser un kanban.

            Nous lui avons répondu que oui puisque cela fait partie intégrante de la gestion agile qui nous est demandé.//
            Il nous a par la suite présenter comment les membres du projet utilise leur kanban.
        \end{subsection}

        \begin{subsection}{Connaisances des besoins techniques des AMO}
            Pablo Seban a demandé si les AMO était au courant des fonctionnalités à implémenter et où les implémenter.

            Ces derniers ont rappelé qu'en soit, le rôle des AMO n'est pas technique et que cette connaissance n'était pas 
            utile.
        \end{subsection}

        \begin{subsection}{Quiproquo sur un besoin}
            Pablo Seban nous a demandé de reclarifier le troisième besoin cité dans la présentation car ce sont les 
            documentation que l'on souhaite visualiser et pas les contraintes en soit.

            Nous lui avons répondu que ce besoin avait bien été capturé mais que c'était une erreur de la présentation.
        \end{subsection}

        \begin{subsection}{Précision de la version de VueJS}
            Florian Azizen nous a demandé si nous connaisions la version de VueJS a utilisé, et plus tard cela nous a amené a
            demander quelle était la syntaxe a utilisé, VueJs permettant deux syntaxes.

            Nous lui avons répondu que pour nous c'était la version 3, ce qui était bien le cas. La discussion qui a suivi a 
            précisé qu'il fallait utiliser la \emph{Composition API}.
        \end{subsection}

    \end{section}

    \begin{section}{Actions issues de la réunion}
        \begin{itemize}
            \item Révision de la note de cadrage
            \begin{itemize}
                \item Suppression du besoin de documentation et ce qui en découle (livrable, planning)
                \item Précision de l'API a utilisé dans les contraintes
            \end{itemize}

            \item Prévision de réunion début janvier pour clarifier le besoin du template
        \end{itemize}
        
    \end{section}

    \begin{section}{Décisions issues de la réunion}
        \begin{itemize}
            \item Suppression du besoin de documenter les contraintes
            \item Choix de l'hébergeur de version et du workflow associé
            \item Choix du kanban
        \end{itemize}
        
    \end{section}

\end{document}