\documentclass[]{article}

\usepackage[utf8]{inputenc}
\usepackage[french]{babel}
\usepackage[]{graphicx}

\title{Note de cadrage}
\author{
    Théo Delmas\\
    Lauric Teysseyre\\
    Pierre-Louis Renon\\
    Julien Wattier\\
    \\
    Université Paul Sabatier\\
    Master Informatique 1\\
   } 
\date{}


\begin{document}
    \maketitle
    \newpage
    \tableofcontents
    \newpage

    \begin{section}{Le projet}
        \begin{subsection}{Contexte}
            FlopEDT! Est une application permettant de créer un emploi du temps satisfaisant une série de contraintes basé sur 
            la programmation linéaire. Cependant l’utilisation de ces contraintes est relativement technique.

            De plus à l’heure actuelle le couplage entre le back-end et le front-end est trop important.
        \end{subsection}

        \begin{subsection}{Objectifs}
            L’objectif est de créer une interface intuitive permettant de rechercher des contraintes à ajouter à la génération 
            d’un emploi du temps, tout en y intégrant la documentation des contraintes.
        \end{subsection}

        \begin{subsection}{Parties prenantes}
            \begin{subsubsection}{Maître d'ouvrage}
                Besoin : Obtenir une interface intuitive pour son produit permettant aux usagers finaux non techniciens de 
                facilement trouver et ajouter des contraintes. Avoir un suivi transparent sur le projet. 

                Degré d’implication : Fort
            \end{subsubsection}

            \begin{subsubsection}{Contributeurs}
                Besoin : Obtenir des livrables maintenables.

                Degré d’implication : Faible
            \end{subsubsection}

            \begin{subsubsection}{Fournisseur}
                Besoin : Satisfaire au maximum les besoins des autres partie prenantes.

                Degré d’implication : Fort
            \end{subsubsection}

            \begin{subsubsection}{Utilisateur final}
                Besoin : Utiliser efficacement l’application.

                Degré d’implication : Faible
            \end{subsubsection}

            \begin{subsubsection}{Assistant de maîtrise d’ouvrage}
                Besoin : Avoir un suivi transparent sur le projet.

                Degré d’implication : Moyen
            \end{subsubsection}

            \begin{subsubsection}{Professeurs}
                Besoin : Obtenir des documents de management de qualité.

                Degré d’implication : Fort
            \end{subsubsection}
        \end{subsection}
    \end{section}

    \begin{section}{Le Périmètre du projet}
        \begin{subsection}{Objectifs}
            Ajout d’une interface de visualisation dynamique de la documentation d’une contrainte.

            Ajout d’une interface de recherche des contrainte pour les ajouter à l’emploi du temps.

            Documenter les contraintes existantes. (Secondaire)
        \end{subsection}

        \begin{subsection}{Hypothèses}
            La maintenance sera effectué par des contributeurs.

            La documentation des contraintes sera en partie fournie.
        \end{subsection}

        \begin{subsection}{Contraintes}
            La documentation sera réalisée en markdown.

            Les interfaces seront réalisées à l’aide de VueJS.
        \end{subsection}

        \begin{subsection}{Livrables}
            \begin{itemize}
                \item Template de documentation.
                \item Interface de visualisation.
                \item Interface de recherche.
            \end{itemize}

            Dans le cas où les produits ci-dessus seraient livrés avant la fin du projet : 
            documentations des contraintes existantes.
        \end{subsection}

        \begin{subsection}{Ressources et délais}
            L’équipe des fournisseurs se compose de quatre membre à compétence équivalente, disponible 16 heures par semaine 
            pendant 16 semaines à partir de début janvier jusque fin avril.           
        \end{subsection}

        \begin{subsection}{Procédure de développement}
            L’équipe des fournisseurs applique la méthode agile. De fait elle s’organisera en sprint au bout desquels se tiendra 
            une cérémonie présentant au client le travail effectuer durant le sprint et prenant en compte ses retours sur le 
            travail réalisé ainsi que ses demande d’évolution.

            La conception des interfaces se fera selon une conception centrée utilisateur.

            La vérification sera donc effectuer au fur et à mesure du projet.

            Un github workflow sera utilisé.
        \end{subsection}

        \begin{subsection}{Procédure de livraison}
            Les livraisons seront identifiée via des releases sur l’outil de version.

            Nous devrons nous assurer que les interfaces sont intégrable à l’existant.
        \end{subsection}

        \begin{subsection}{Procédure d’évolution du périmètre}
            En fin de sprint le client pourra effectuer une demande de modification du périmètre qui sera étudié
            par l’équipe de fournisseur. Cette dernière ré-évaluera son planning en prenant en compte
            d’éventuels apprentissage, puis soumettra le planning au client. Si ce dernier accepte l’équipe de
            fournisseur mettra à jour l’ensemble des documents du projet et entamera un nouveau sprint.

            L’ensemble des versions des documents du projet devront être conservé.
        \end{subsection}
    \end{section}

    \begin{section}{Procédures de management}
        L’ensemble des documents produits seront soumis à une vérification de la part de nos enseignants et
        corriger en dehors du temps alloué au projet dans les plus bref délais pour se conformer au niveau
        de qualité attendu.

        Ces même professeurs définiront le niveau de qualité attendu.
    \end{section}

    \begin{section}{Jalons}
        L’analyse préliminaires des livrables montre les dépendances suivantes :

        Une première proposition de macro-planning est la suivante :

        \begin{figure}[h]
            \begin{center}
                \includegraphics[scale=0.15]{Gantt}
            \end{center}
        \end{figure}
    \end{section}

    \begin{section}{Risques}
        \begin{subsection}{Retard sur jalon}
            Prévention : Une estimation large des tâches sera effectuer pour éviter de mauvaise surprise pour le
            client.

            Risque d’occurrence : Moyen

            Niveau d’impact : Moyen

            Protocole en cas de déclenchement :\\
            La présentation des tâches est décalée d’un sprint. Le planning est revue et le client prévenu.
        \end{subsection}

        \begin{subsection}{Absence temporaire d’un fournisseur}
            Prévention : Puisque l’ensemble des développeurs ont concrètement les mêmes compétences, il est
            inutile de prévoir des chaînes de back-up.

            Risque d’occurrence : Moyen

            Niveau d’impact : Moyen

            Protocole en cas de déclenchement :\\
            Les fournisseurs restant effectueront une réunion d’urgence pour déterminer si les tâches assignés à
            l’absent dépassent leurs intervalles dans le macro-planning. Si c’est le cas une réallocation des
            tâches sera effectuer afin de limiter voir annuler les inter-blocages lié à l’ordonnancement des tâche.

            Autrement rien ne change, les tâches seront seulement repoussées dans le planning.
        \end{subsection}

        \begin{subsection}{Absence définitive d’un fournisseurs}
            Prévention : Puisque l’ensemble des développeurs ont concrétement les mêmes compétences, il est
            inutile de prévoir des chaînes de back-up.

            Risque d’occurrence : Faible

            Niveau d’impact : Fort

            Protocole en cas de déclenchement :\\
            Les fournisseurs restant effectueront une réunion d’urgence pour ré-évaluer le planning. Dans le cas
            où ce nouveau planning dépasserait la date de livraison, une réunion avec les clients sera effectuer
            pour les informer de la situation. Autrement le client sera juste notifier d’une modification de la
            composition et recevra le nouveau planning.
        \end{subsection}

        \begin{subsection}{Absence temporaire d’un client principal}
            Prévention : Mise en place d’une chaîne de back-up pour assurer le rôle de client principal.

            Risque d’occurrence : Moyen.

            Niveau d’impact : Moyen

            Protocole en cas de déclenchement :\\
            Les fournisseurs effectueront une réunion d’urgence pour évaluer si cette absence modifie le
            planning, notamment au niveau des validations par le client.\\
            Toutes les communications et cérémonies seront redirigés vers un client secondaire désigné par le
            client principale. Dans le cas où il n’y a aucun client secondaire, les cérémonies ne seront pas
            assuré. Les communications seront mise en tampon. Le travail de l’équipe de fournisseur continuera
            normalement jusqu’au jalon de validation qui seront reporté
        \end{subsection}

        \begin{subsection}{Absence définitive d’un client}
            Prévention : S’assurer qu’il y ait plus d’un client principal

            Risque d’occurrence : Faible.

            Niveau d’impact : Catastrophique

            Protocole en cas de déclenchement :\\
            L’ensemble des communications et cérémonies se dérouleront avec le reste des clients principaux.
            Dans le cas où tous les clients principaux sont indisponibles, le projet sera annulé faute de client.
            
        \end{subsection}

        \begin{subsection}{Manque de maîtrise d’un fournisseur sur VueJS}
            Prévention : Formation à vuejs

            Risque d’occurrence : Faible.

            Niveau d’impact : Moyen

            Protocole en cas de déclenchement :\\
            L’équipe de fournisseur effectuera une réunion d’urgence pour décider si le manque est trop
            important ou non. Si ce n’est pas le cas les tâches du fournisseur défaillant seront revues pour qu’ils
            effectue les plus simple ou qu’il dispose de plus de temps afin qu’il puisse se documenter en même
            temps qu’il produit. Sinon, le fournisseur défaillant passera en peer-programing.\\
            Quoi qu’il arrive le planning sera revu et le client informé de la situation.
        \end{subsection}

        \begin{subsection}{Perte de donnée}
            Prévention : Utilisation d’un logiciel de gestion de version décentralisée.

            Risque d’occurence : Faible

            Niveau d’impact : Fort.

            Protocole en cas de déclenchement : Aucun, gérer par la prévention.
        \end{subsection}
    \end{section}

    \begin{section}{Communication}
        Nous ne sommes pas en contact direct avec les contributeurs. La seule interaction se fera au travers
        de la documentation.

        L’ensemble de communication des fournisseurs se feront sur un serveur discord car ce dernier est
        facilement organisable. Ils effectueront des daily meetings permettant à chacun de savoir ce que fait
        l’autre.

        Le client sera informé du travail au travers de revues de backlog avec tous les fournisseurs, au
        travers du serveur discord FlopEDT! qui centralise déjà toute les communication du projet. Ce
        discord sera également utilisé pour contacter les clients au besoin hors des cérémonies usuelles.

        L’utilisation d’un outil de version permettra aux différentes partie prenante de suivre l’avancée du
        projet.
        
    \end{section}
\end{document}