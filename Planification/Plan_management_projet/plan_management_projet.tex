\documentclass[]{article}

\usepackage[utf8]{inputenc}
\usepackage[french]{babel}
\usepackage[]{graphicx}
\usepackage[]{hyperref}

\title{Plan de management projet}
\author{
    Théo Delmas\\
    Lauric Teysseyre\\
    Pierre-Louis Renon\\
    Julien Wattier\\
    \\
    Université Paul Sabatier\\
    Master Informatique 1\\
   } 
\date{}


\begin{document}
\maketitle
\newpage
\tableofcontents
\newpage

\begin{section}{Gestion des parties prenantes}
 L'ensemble des parties prenantes est listée dans le \href{../Parties_prenantes/Parties_prenantes.pdf}{fichier des parties prenantes}.

\end{section}

\begin{section}{Gestion des besoins}
 L'ensemble des besoins sont listés sous forme d'user story dans le \href{https://wekan.flopedt.org/b/HsRkBw5rbmQt5PQet/catalogue}{kanban dédié }.

 \begin{subsection}{Capture des besoins}
     Les besoins sont capturé au travers de réunions dédiées tenues avant de commencer un jalon.
     Par la suite elle sont étoffées grâce à des user story qui sont pas la suite insérées dans le kanban.
 \end{subsection}

 \begin{subsection}{évolution des besoins}
    Lors des revue de sprint, les parties prenantes présentes peuvent soumettre des demandes d'évolution.

 \end{subsection}

\end{section}
\end{document}