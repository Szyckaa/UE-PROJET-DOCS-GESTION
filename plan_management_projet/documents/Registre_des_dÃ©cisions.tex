\documentclass[]{article}

\usepackage[T1]{fontenc}
\usepackage[utf8]{inputenc}
\usepackage[french]{babel}
\usepackage[]{graphicx}
\usepackage[]{hyperref}

\title{Registre des décisions}
\author{
    Théo DELMAS\\
    Lauric TEYSSEYRE\\
    Pierre-Louis RENON\\
    Julien WATTIER\\
    \\
    Université Paul Sabatier\\
    Master Informatique 1\\
   } 

\begin{document}
\maketitle
\newpage
\tableofcontents
\newpage

\begin{section}{Objectif du document}
 Ce document référence l'ensemble des décisions ayant amendées le plan de projet.
\end{section}

{
\setlength{\parindent}{0pt} %Retire les alinéas
\begin{section}{Liste des décisions}
 \begin{subsection}{Systématisation de l’utilisation de liveshare}
     Description : Ajout d’un paragraphe concernant l’utilisation de liveshare pour débugger les autres développeurs dans le plan de communication.

     Contexte : Le débogage des pairs était compliqué car pour voir le code il fallait streamer son IDE et les autres développeurs ne pouvaient donc pas voir tout le code. D’autre part des propositions de peer-programming avaient été émise.

     Date : 15/03/2023
 \end{subsection}
 \begin{subsection}{Écriture de rapport de réunion}
     Description : Ajout d’un paragraphe précisant que les revues de sprint sont enregistrées et que des rapports de réunion sont produits ultérieurement.

     Contexte : La seconde version du plan projet nous a poussé à revoir notre organisation et notamment à avoir une trace écrite des revues de sprint qui sont des réunions majeures pour ce projet.

     Date : 16/03/2023
 \end{subsection}
 \begin{subsection}{Section sur la documentation}
     Description : Ajout d’une section concernant la documentation dans la gestion de la qualité.

     Contexte : Lors d’une revue de sprint la question de la clôture du projet a été abordée et notamment la maintenance. Il a été décidé de bien documenter le code source.

     Date : 29/03/2023
 \end{subsection}
 \begin{subsection}{Arrêt de production sur le système de recherche}
     Description : L’ensemble des informations relatives à la production d’un système de recherche des contraintes est retiré du rapport.

     Contexte : Faute de temps pour continuer le projet, l’équipe s’est mis d’accord avec le client pour ne pas produire le système de recherche de contrainte et de se consacrer sur l’affichage des documentations.

     Date : 29/03/2023
 \end{subsection}
 \newpage
 \begin{subsection}{Demande de revue pour la livraison finale}
     Description : Ajout d’une section précisant la livraison finale dans la gestion des livrables.

     Contexte : Le client a demandé à un collaborateur du projet qualifié sur le front-end pour effectuer une revue de code avant la clôture du projet.

     Date : 29/03/2023
 \end{subsection}

 \begin{subsection}{Rédaction des livrables de gestion en latex}
    Description : Ajout d'indication précisant que les documents sont rédigés en Latex et qu'ils sont revus à l'aide de git.

    Contexte : Le suivi de modification ; et donc la revue ; était compliqué en rédigeant sous libre office. Pour régler ce problème il a été décidé de rédiger les documents de gestion en latex, ce qui permet de pleinement tiré profit des fonctionnalités de git afin d'effectuer des revues.

    Date : 09/04/2023
\end{subsection}

\begin{subsection}{Création d'un registre des faits marquants}
    Description : Ajout d'indication d'une section pour le suivi du projet.

    Contexte : Aucun document ne listait les différents événements qu'a rencontrés le projet. Pour comprendre un peu plus le contexte un document dédié devait donc lister ces événements.

    Date : 09/04/2023
\end{subsection}
\end{section}
}
\end{document}