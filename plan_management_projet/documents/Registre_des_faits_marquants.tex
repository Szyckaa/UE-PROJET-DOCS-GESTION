\documentclass[]{article}

\usepackage[utf8]{inputenc}
\usepackage[french]{babel}
\usepackage[]{graphicx}
\usepackage[]{hyperref}
\usepackage{tabularx}
\usepackage{float}
\usepackage[table]{xcolor}
\usepackage[T1]{fontenc}

\title{Registre des faits marquants}
\author{
    Théo DELMAS\\
    Lauric TEYSSEYRE\\
    Pierre-Louis RENON\\
    Julien WATTIER\\
    \\
    Université Paul Sabatier\\
    Master Informatique 1\\
   } 

\begin{document}
\maketitle
\newpage
\tableofcontents
\newpage

\begin{section}{Objectif du document}
 Ce document référence les événements s'étant produits durant chaque sprint.
\end{section}

{
\setlength{\parindent}{0pt} %Retire les alinéas
\begin{section}{Faits marquant}
    \begin{subsection}{Sprint 0}
        \begin{table}[H]
            \caption{Évenements du sprint 0}
            \begin{tabularx}{\columnwidth}{|X|X|}
                \hline
                \cellcolor[HTML]{17FF00}Positif & \cellcolor[HTML]{FF2D00}Négatif\\
                \hline
                Un composant VueJS nommé VueShowdown permettant de convertir automatiquement un fichier markdown en HTML associé et l'afficher a été découvert.& La mise en place de l'environnement de travail a d'abords été installé sous docker mais il s'est révélé que la configuration Docker ne fonctionnait pas. L'environnement de travail a été finalement installé sur des machines virtuelles.\\
                \hline
                Le prototypage d'IHM a été validé par le client dès le premier jet. & La configuration de l'environnement de travail sous WSL a posé quelques problèmes, notamment au niveau de la hot-reload de nodes.\\
                \hline
                & La structure du projet est assez peu documentée. Cela nous a pris de temps pour la comprendre.\\
                \hline
                & Le client a oublié la revue de sprint. Cette dernière a finalement été tenue avec du retard mais en accélérée et l'équipe n'a pas pensé à lancer l'enregistrement de la réunion.\\
                \hline
            \end{tabularx}
        \end{table}
    \end{subsection}

    \begin{subsection}{Sprint 1}
        \begin{table}[H]
            \caption{Évenements du sprint 1}
            \begin{tabularx}{\columnwidth}{|X|X|}
                \hline
                \cellcolor[HTML]{17FF00}Positif & \cellcolor[HTML]{FF2D00}Négatif\\
                \hline
                 & Les liens d'images dans les documentations étaient relatifs au localhost et ne seraient donc pas passer en environnement de production. Il a fallu ajouter un système altérant les documentations soumises.\\
                \hline
            \end{tabularx}
        \end{table}
    \end{subsection}

    \begin{subsection}{Sprint 2}
        \begin{table}[H]
            \caption{Évenements du sprint 2}
            \begin{tabularx}{\columnwidth}{|X|X|}
                \hline
                \cellcolor[HTML]{17FF00}Positif & \cellcolor[HTML]{FF2D00}Négatif\\
                \hline
                 & Le client nous a indiqué que l'ensemble de leurs systèmes utilisait la convention que si une collection était vide alors elle désignait en fait l'ensemble des instances de la classe. Cela nous oblige à revoir certaines logiques.\\
                \hline
            \end{tabularx}
        \end{table}
    \end{subsection}

    \begin{subsection}{Sprint 3}
        \begin{table}[H]
            \caption{Évenements du sprint 3}
            \begin{tabularx}{\columnwidth}{|X|X|}
                \hline
                \cellcolor[HTML]{17FF00}Positif & \cellcolor[HTML]{FF2D00}Négatif\\
                \hline
                L'équipe à découvert une extension VSCode nommée Liveshare qui permet de faire du peer-programming. Cela facilite le débogage. & L'équipe connaît une baisse de motivation qui la rend moins productive. \\
                \hline
                Une rétrospective a été tenue en cours d'agilité et nous a permis de nous rendre compte que nos daily meeting était bien trop longue. & \\
                \hline
                L'équipe s'est rendu compte que la production du système de recherche ne serait pas tenable dans les délais et a donc demandé son annulation afin de se focaliser sur l'affichage des documentations. Le client a accepté la demande. & \\
                \hline
            \end{tabularx}
        \end{table}
    \end{subsection}

    \begin{subsection}{Sprint 4}
        \begin{table}[H]
            \caption{Évenements du sprint 4}
            \begin{tabularx}{\columnwidth}{|X|X|}
                \hline
                \cellcolor[HTML]{17FF00}Positif & \cellcolor[HTML]{FF2D00}Négatif\\
                \hline
                 & L'organisation de la revue de sprint, qui doit être suivie d'une revue de code général avec les mainteneurs du projet, pose des problèmes d'organisation à cause d'emplois du temps incompatibles. \\
                \hline
            \end{tabularx}
        \end{table}
    \end{subsection}

    \begin{subsection}{Sprint 5}
        \begin{table}[H]
            \caption{Évenements du sprint 5}
            \begin{tabularx}{\columnwidth}{|X|X|}
                \hline
                \cellcolor[HTML]{17FF00}Positif & \cellcolor[HTML]{FF2D00}Négatif\\
                \hline
                Tenu d'une rétrospective globale au projet afin de rédiger le bilan. & \\
                \hline
            \end{tabularx}
        \end{table}
    \end{subsection}

    \begin{subsection}{Sprint 6}
        \begin{table}[H]
            \caption{Évenements du sprint 6}
            \begin{tabularx}{\columnwidth}{|X|X|}
                \hline
                \cellcolor[HTML]{17FF00}Positif & \cellcolor[HTML]{FF2D00}Négatif\\
                \hline
                 & \\
                \hline
            \end{tabularx}
        \end{table}
    \end{subsection}
\end{section}
}

\end{document}