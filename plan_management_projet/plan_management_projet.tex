\documentclass[]{article}

\usepackage[utf8]{inputenc}
\usepackage[french]{babel}
\usepackage[]{graphicx}
\usepackage[]{hyperref}

\title{Plan de management projet}
\author{
    Théo Delmas\\
    Lauric Teysseyre\\
    Pierre-Louis Renon\\
    Julien Wattier\\
    \\
    Université Paul Sabatier\\
    Master Informatique 1\\
   } 
\date{}


\begin{document}
\maketitle
\newpage
\tableofcontents
\newpage

\begin{section}{Objectif du document}

\end{section}

\begin{section}{Contexte}
    \begin{subsection}{Contexte}
        FlopEDT! est une application permettant de créer un emploi du temps satisfaisant une série de contraintes basées sur 
        la programmation linéaire. Cependant la gestion actuelle de ces contraintes est relativement technique et peu documentée.

        De plus les principaux collaborateurs du projet ont initié un découplage entre le back-end et le front-end, les deux parties étant actuellement gérées par un unique serveur.
    \end{subsection}

    \begin{subsection}{Objectifs}
        L’objectif est de créer une interface intuitive permettant de rechercher des contraintes à ajouter à la génération 
        d’un emploi du temps, tout en y intégrant la documentation des contraintes, le tout en respectant le contexte de découplage front/back de l'application.
    \end{subsection}
\end{section}

\begin{section}{Plans}
\begin{subsection}{Scope management plan}

\end{subsection}
\begin{subsection}{Requirement management plan}

\end{subsection}
\begin{subsection}{Schedule management plan}

\end{subsection}
\begin{subsection}{Quality management plan}

\end{subsection}
\begin{subsection}{Ressource management plan}

\end{subsection}
\begin{subsection}{Communication management plan}

\end{subsection}
\begin{subsection}{Risk management plan}

\end{subsection}
\begin{subsection}{Stakeholder engagement plan}

\end{subsection}
\begin{subsection}{Change management plan}

\end{subsection}
\begin{subsection}{Configuration management plan}

\end{subsection}
\begin{subsection}{Scope baseline}

\end{subsection}
\begin{subsection}{Schedule baseline}

\end{subsection}
\begin{subsection}{Performance measurement baseline}

\end{subsection}
\begin{subsection}{Project life cycle description}

\end{subsection}
\begin{subsection}{Developpment approach}

\end{subsection}
\begin{subsection}{Management reviews}

\end{subsection}
\end{section}
\end{document}