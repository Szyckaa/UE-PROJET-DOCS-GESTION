\documentclass[]{article}

\usepackage[utf8]{inputenc}
\usepackage[french]{babel}
\usepackage[]{graphicx}
\usepackage[]{hyperref}

\title{Plan de management projet}
\author{
    Théo Delmas\\
    Lauric Teysseyre\\
    Pierre-Louis Renon\\
    Julien Wattier\\
    \\
    Université Paul Sabatier\\
    Master Informatique 1\\
   } 
\date{}


\begin{document}
\maketitle
\newpage
\tableofcontents
\newpage

\begin{section}{Objectif du document}
 Ce document a pour objectif d'exposé les méthodes de management appliqué à ce projet.
\end{section}

\begin{section}{Contexte}
 \begin{subsection}{Contexte}
     FlopEDT! est une application permettant de créer un emploi du temps satisfaisant une série de contraintes basées sur
     la programmation linéaire. Cependant la gestion actuelle de ces contraintes est relativement technique et peu documentée.

     De plus les principaux collaborateurs du projet ont initié un découplage entre le back-end et le front-end, les deux parties étant actuellement gérées par un unique serveur.
 \end{subsection}

 \begin{subsection}{Objectifs}

     L’objectif est de créer une interface intuitive permettant de rechercher des contraintes à ajouter à la génération
     d’un emploi du temps, tout en y intégrant la documentation des contraintes, le tout en respectant le contexte de découplage front/back de l'application.
 \end{subsection}
\end{section}

\begin{section}{Plans}
 \begin{subsection}{Scope management plan}

     \begin{subsubsection}{Définition de la scope baseline}
         Les produits du projet ont été définis lors de la kick-off meeting et sa préparation.

         Ils sont référencé dans la scope baseline.

         Au fur et à mesure du projet, si de nouveau produits apparaissent, ils seront ajoutés à la WBS via une demande de changement interne.
     \end{subsubsection}

     \begin{subsubsection}{Validation de la scope baseline}
         Lorsque le document sera rédigé, une demande de validation sera envoyé aux parties prenantes.
     \end{subsubsection}

 \end{subsection}

 \begin{subsection}{project baseline}
     Description du cadre

     Liste des livrables == WBS en agile == sprint

     Critère d'acceptation

     Exclusion

 \end{subsection}

 \begin{subsection}{Requirement management plan}
     Les besoins sont suivis grâce à un backlog produit. Celui-ci se décompose en deux parties. La première défini les principaux produits du projet et permet aux parties prenante de comprendre sur quelle produit l'équipe travail actuellement. Les produits ayant été défini lors de la kick-off meeting.

     \begin{subsubsection}{Backlog produit}

         La seconde est un backlog lié au produit en cours. L'ensemble des besoins y sont retranscrit et décrit grâce à des user stories sur lesquelles l'équipe fournisseur adjoint une liste de tâche a effectuer afin de satisfaire la user story.

         Le backlog se divise en 5 colonnes :
         \begin{itemize}
             \item Idée floue : La user story nécessite d'être raffinée. Cette colonne permet de noter informellement les besoins du clients
             \item A faire : La user story a été défini mais pas choisie pour le sprint en cours.
             \item En cours : La user story a été choisie pour le sprint en cours.
             \item A tester : La user story est satisfaite mais nécessite une validation.
             \item Ok : La user story est terminée et testée.
         \end{itemize}

         Chaque user story est priorisée via sa position dans son couloir, et relativement aux besoins du clients.

     \end{subsubsection}

     \begin{subsubsection}{Capture des besoins}
         Une première partie des besoins ont été capturé lors de la kick-off meeting et sa préparation.

         Par la suite les besoins seront capturés et défini précisement au travers des revues de sprint.

         Si il apparait que des besoins sont attachés à un produit non identifié, alors la WBS sera mise à jour.
     \end{subsubsection}

     \begin{subsubsection}{Gestion des changements dans les besoins}
         Les changements dans la liste des tâches à effectuer pour chaque besoin n'est pas tracée.
         En revanche l'ensemble des besoins annulés sont recupérables dans l'archive du kanban.

         Chaque développeur peut manipuler le kanban sans autorisation particulière.
     \end{subsubsection}

 \end{subsection}

 \begin{subsection}{Schedule management plan}

 \end{subsection}

 \begin{subsection}{Quality management plan}

 \end{subsection}

 \begin{subsection}{Ressource management plan}
     Ce projet est sans ressources matériels.

     \begin{subsubsection}{Entrainement de l'équipe}
         L'entrainement de l'équipe se fait au fur et à mesure du développement, sans activité dédiée.
     \end{subsubsection}


 \end{subsection}

 \begin{subsection}{Communication management plan}

 \end{subsection}

 \begin{subsection}{Risk management plan}
    \begin{subsubsection}{Référencement des risques}
        La liste des risques et opportunité est listée dans le \href{./documents/risques.pdf}{registre des risques et opportunité}.
        Chaque risque y est décrit par les champs suivants : 
        % \begin{itemize}
        %     % TODO
        % \end{itemize}
    \end{subsubsection}
 \end{subsection}

 \begin{subsection}{Stakeholder engagement plan}

     \begin{subsubsection}{Procédé d'identification des parties prenantes}
         Une première phase d'identification a été ménée lors de l'élaboration de la charte de projet.

         Par la suite, les parties prenantes seront potentiellement identifiées au fur et à mesure que des besoins apparaissent. Auquel cas l'équipe fournisseur émettra une demande de changement.
     \end{subsubsection}

     \begin{subsubsection}{Référencement des parties prenantes}
         La liste des parties prenante est listée dans le \href{./documents/Registre_des_parties_prenantes.pdf}{registre des parties prenantes}.
         Chaque partie prenante y est décrite par son rôle, ses besoins principaux, son niveau d'engagement, ainsi que son niveau d'influence sur le projet.

         Le niveau d'engagement représente à quel point la partie prenante est impliqué dans le projet. Il influence notamment le niveau de communication qui doit être maintenu avec la partie prenante.
         Les niveau d'engagement possible sont les suivants :
         \begin{itemize}
             \item Ignorant : La partie prenante n'est pas au courant du projet.
             \item Résistant : La partie prenante est au courant du projet et s'y oppose.
             \item Neutre : La partie prenante est au courant du projet mais n'y accorde pas d'importance.
             \item Favorable : La partie prenante est au courant du projet, et le supporte.
             \item Impliqué : La partie prenante est au courant du projet, et s'implique dans sa réalisation
         \end{itemize}

         Les niveau d'influence possibles sont les suivants et représente à quel point la partie prenante peut imposer son pouvoir de décision sur le projet.
         \begin{itemize}
             \item Faible : La partie prenante n'a pas ou peu de pouvoir de décision sur le projet.
             \item Moyen : La partie prenante peut influencer dans une certaine mesure le projet.
             \item Fort : La partie prenante a un fort pouvoir de décision.
         \end{itemize}

     \end{subsubsection}
 \end{subsection}

 \begin{subsection}{Change management plan}
    \begin{subsubsection}{Référencement des requêtes de changement}
        L'ensemble des requêtes de changement sont listés dans le document \href{./documents/Requête_de_changements.pdf}{requêtes de changement}.
        Elles y sont décrite par les champs suivants : 
        \begin{itemize}
            \item Titre de la requête : Description succinte de la requête.
            \item Date d'émission.
            \item Émetteur de la requête.
            \item Domaine de connaissance impacté.
            \item Description : quelle modification est demandé.
            \item Date de traitement.
            \item Responsables de la requête : Qui a appliqué la réponse à cette requête.
            \item Réponse : ensemble des modifications ayant été effectué après traitement.
        \end{itemize}
    \end{subsubsection}

    \begin{subsubsection}{Traitement des requêtes de changement}
        Les requêtes sont traitées en priorité par l'équipe de fournisseur, idéalement juste après leur émission.
        Le responsable de la requête veillera a completé les champs de la requête traité.
    \end{subsubsection}

    \begin{subsubsection}{Émission des requêtes de changement}
        Tout événement donnant lieu à une modification des informations relatives à ce projet produira une demande de changement.
        De fait certaines partie prenante de part leur niveau d'influence vont naturellement produire ce genre d'évenement.
        L'équipe fournisseur veillera donc à produire les requête associée.

        Dans le cas d'événement impactant plusieurs domaines de connaisances, une requête par domaine devra être produite.
    \end{subsubsection}
 \end{subsection}

 \begin{subsection}{Configuration management plan}
     Les produits de ce projet seront tracés à l'aide de git, sur le \href{https://framagit.org/flopedt/FlOpEDT}{dépôt officiel de FlopEDT}.
     D'autre part le plan de projet est également tracé à l'aide de git et consultable sur \href{https://github.com/Szyckaa/UE-PROJET-DOCS-GESTION}{le dépôt distant}.

     Dans les deux cas, chaque développeur dispose des droits et responsabilités sur les actions qu'il entreprend relativement à ces dépôts.
 \end{subsection}

 \begin{subsection}{Scope baseline}

 \end{subsection}

 \begin{subsection}{Schedule baseline}

 \end{subsection}

\end{section}
\end{document}